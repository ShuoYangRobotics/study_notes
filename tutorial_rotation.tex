% dependencies: xelatex, xecjk package,wenquanyi中文字体,也可以设置其他的中英文字体
%!TEX program = xelatex
\documentclass[11pt]{article}
\usepackage[boldfont]{xeCJK}
\usepackage{amsmath}
\usepackage{esvect} %一个更好的矢量描述工具
%\setmainfont{Courier New} % 设置英文衬线字体
% \setmonofont{} % 设置英文等宽字体,等宽英文字体大全:http://zh.wikipedia.org/wiki/%E7%AD%89%E5%AE%BD%E5%AD%97%E4%BD%93
% \setsansfont{} % 设置英文无衬线字体
\setCJKmainfont{WenQuanYi Micro Hei} % 设置缺省中文字体
%\setCJKfamilyfont{WenQuanYi Micro Hei} % 与setCJKmainfon t等同,http://bbs.ctex.org/forum.php?mod=viewthread&tid=51057
\parindent 2em   %段首缩进
 
\begin{document}
%%%%%%%%%%%%%%%%%%%%%%%%%%%%%%%%%%%%%%%%%%%%
%    这里是文档的开头                                                                                                                      %
%%%%%%%%%%%%%%%%%%%%%%%%%%%%%%%%%%%%%%%%%%%%
\title{多旋翼飞行器动力学建模中的数学知识}
\author{杨硕}

\maketitle
这篇教程围绕多旋翼飞行器的动力学模型展开介绍,对旋转表示法、角速度、惯量、欧拉角、四元数、欧拉方程等概念进行深入讲解。

一般来说,研究多旋翼飞行器的工程师和科学家们都是电子、机械、计算机科学背景,他们除了非常基本的数学物理课程之外,可能并没有系统地学习基本的经典力学知识和代数知识。
然而按照正统的数学和物理课程学习相关知识又会花费过多的时间,因此我认为通过多旋翼飞行器控制中对数学要求最高的部分:动力学建模,穿插介绍相关的数学物理知识,是一个比较方便整理思路并且易于理解的学习方法。
因为每一个数学或者物理的知识点被介绍出来之后,都会立即应用到实际的工程问题中,读者很直观地看到如何用数学和物理知识指导实际需求。

本教程尽量用明白清晰的语言介绍各个知识点,同时尽量不跳过任何公式推导的中间过程。有些地方可能会显得行文冗长多余,数学功底好的读者可以略去公式前后文的解释。

\section{基本符号定义和坐标系定义}
本节介绍一些基本的符号定义以及飞行器控制中常用的坐标系定义,教程剩余章节中将大量使用本节中的符号。
\subsection{基本符号定义}
$a$,小写字母表示标量

$\vv{v}$,字母上加箭头表示矢量。很多教材会使用黑体字母,如$\bf{v}$,来表示矢量,这样一方面容易和矩阵混淆,一方面容易和标量混淆。虽然箭头在有些公式中显得过于复杂,但是能够方便地让读者时刻记得公式中符号的数学意义。

$\dot{a}$,代表对标量求导,我们默认标量$a$是以$t$为自变量的函数$a(t)$,因此就把$t$隐去以简化符号。

$\dot{\vv{v}}$,代表对矢量求导,我们默认矢量$\vv{v}$中每一个元素$v_i$都是以$t$为自变量的标量函数

$\bf{A}$,黑体大写字母表示矩阵

$\bf{I}_{3x3}$,有时会通过下标的方式指明矩阵的尺寸。常见的例子是用来指明单位矩阵(Identity Matrix)$\bf{I}$的尺寸,比如3乘3的单位矩阵表示为$\bf{I}_{3x3}$。

$\bf{A}^{-1}$,对矩阵求逆

$\bf{A}^T$,对矩阵求转秩

$\bar{\bf{q}}$,表示四元数

$\bf{R}_{ab}$,表示旋转矩阵,$a$和$b$代表两个\textbf{原点相交}的坐标系,旋转矩阵将$b$坐标系中表示的矢量旋转到$a$坐标系中

$\bar{\bf{q}}_{ab}$,表示旋转四元数,下标的含义与旋转矩阵中的定义相同。

$\lfloor \vv{\omega} \ \times \rfloor$,将矢量$\vv{\omega}$变换成反对称矩阵$\lfloor \vv{\omega} \ \times \rfloor$(将会在后面详细介绍)

其他未说明的符号一部分是标准的数学计算符号,另一部分则在教程中定义
\subsection{坐标系定义}
航空航天工程中发展出了描述飞行器在空间中相对于地球的标准表达方式。在描述动力学模型中,我们只需要关心大地坐标系和机体坐标系。
\subsubsection{大地坐标系}
飞行器放置在地面上,启动之后飞行器离开这一地点,但是很多的传感器都是以地面上这一点作为基准点的。为了描述飞行器相对于这一地面基准点的位置和姿态角,我们要在地面上附加一个坐标系,叫做大地坐标系(ground frame)或者地球坐标系(earth frame)。相应地,有时也会将这种坐标系缩写为坐标系$g$或者坐标系$e$。本教程中我们用$e$来指代大地坐标系,$e$会出现在很多公式符号的下标中,特别是旋转矩阵和旋转四元数的下标中。

大地坐标系遵循一般笛卡尔坐标系中坐标轴关系,三个坐标轴的指向与地理方位相关。在基准点上,$x$轴指向北方,$y$轴指向东方,$z$轴指向地心。因此,大地坐标系也被称作“北东地”坐标系(NEG frame)。
\subsubsection{机体坐标系}\label{sec:bodyframe}
飞行器上也需要附加一个坐标系,以便描述飞行器自身的部件位置,以及描述自身的旋转。无论飞行器如何变换自身的姿态,这个坐标系相对于飞行器的位置不会发生变化。机体坐标系的坐标轴关系与一般笛卡尔坐标系相同,只是我们将$x$轴称为“横滚”轴(roll axis)、$y$轴称为“俯仰”轴(pitch axis)、$z$轴称为“偏航”轴(yaw axis)。如下图所示:

本教程中我们用$b$来指代机体坐标系。

\section{旋转矩阵}
在定义坐标系之后,我们所关注的问题是如何描述飞行器相对地面的位置和姿态。对位置来说,如果把飞行器看做质点,那么我们可以很容易地在大地坐标系中描述出飞行器的三维位置,这个位置通常用$\vv{p}$来表示:
$$
\vv{p} = p_1\vv{i}+p_2\vv{j}+p_3\vv{k}
$$
其中$\vv{i}$,$\vv{j}$,$\vv{k}$分别是坐标系$xyz$三轴的单位向量:
$$
\vv{i} = \begin{bmatrix}
1\\0\\0
\end{bmatrix}\ \ \ \ \ 
\vv{j} = \begin{bmatrix}
0\\1\\0
\end{bmatrix}\ \ \ \ \ 
\vv{k} = \begin{bmatrix}
0\\0\\1
\end{bmatrix}
$$
这样我们也可以把单位向量合并起来,把$\vv{p}$简单写成$\vv{p} = \begin{bmatrix}
p_1\\p_2\\p_3
\end{bmatrix}$。在后面的教程中,我们会经常切换两种表示方法以便于解释不同的问题。

在实际中,飞行器并非一个质点,而是\textbf{无数个质点构成的刚体}。为了描述这个刚体相对于大地坐标系的姿态,我们需要引入旋转矩阵的概念。另外对于刚体上不位于质心的任意一点,它在大地坐标系中的位置必须要借助旋转矩阵来描述。
\subsection{通过旋转矩阵描述坐标系之间的变换}
这一节中我们暂时不用飞行器和大地坐标系的术语,而是考虑更一般的情况。

假设空间中有一个坐标系$e$(也可以用$OXYZ$来表示,因为我们用$\vv{OX}$、$\vv{OY}$、$\vv{OZ}$来称呼它的三个轴),有一个刚体处于这个坐标系中,我们在刚体上附加机体坐标系$b$(也可以用$oxyz$来表示,同理它的三个轴是$\vv{ox}$、$\vv{oy}$、$\vv{oz}$)。初始时坐标系$e$和坐标系$b$是重合的。假设刚体因为某些原因变换了姿态,那么坐标系$b$的各轴就不再与坐标系$e$的各轴重合,不过他们的原点依然是重合的。

为了描述变换姿态后坐标系$b$和坐标系$e$之间的关系,我们注意到坐标系$b$的各轴在坐标系$b$中可以用前述的$\vv{i}$、$\vv{j}$和$\vv{k}$来表示,但是如果从坐标系$e$中观察,坐标系$b$的各轴就可以看做坐标系$e$中的普通矢量,那么坐标系$b$的一个轴可以表示成坐标系$e$的三个轴的线性组合。也即
$$
\vv{ox} = r_{11}\vv{OX} + r_{21}\vv{OY} + r_{31}\vv{OZ} 
$$
$$
\vv{oy} = r_{12}\vv{OX} + r_{22}\vv{OY} + r_{32}\vv{OZ} 
$$
$$
\vv{oz} = r_{13}\vv{OX} + r_{23}\vv{OY} + r_{33}\vv{OZ} 
$$
这里的$r_{ij}$表示线性组合的系数,我们可以不失一般性地就把他们当做9个常数。如果我们把这9个数字写成矩阵的形式(注意下标的顺序)
$$
\bf{R} = \begin{bmatrix}
r_{11} & r_{12} & r_{13}\\
r_{21} & r_{22} & r_{23}\\
r_{31} & r_{32} & r_{33}
\end{bmatrix}
$$
注意到$\vv{OX}$、$\vv{OY}$和$\vv{OZ}$其实是坐标系$e$中的$\vv{i}$、$\vv{j}$和$\vv{k}$,我们可以发现
$$
\vv{ox} = r_{11}\vv{OX} + r_{21}\vv{OY} + r_{31}\vv{OZ}  = 
\begin{bmatrix}
r_{11}\\
r_{21}\\
r_{31}
\end{bmatrix} =
\begin{bmatrix}
r_{11} & r_{12} & r_{13}\\
r_{21} & r_{22} & r_{23}\\
r_{31} & r_{32} & r_{33}
\end{bmatrix}
\begin{bmatrix}
1\\
0\\
0
\end{bmatrix} = \bf{R} \vv{i}
$$
同样地
$$
\vv{oy} =   \bf{R} \vv{j}
$$
$$
\vv{oz} =   \bf{R} \vv{k}
$$

在上述的推导中,我们需要注意一个细节:通常我们把任何笛卡尔坐标系的三个轴都称作$\vv{i}$、$\vv{j}$和$\vv{k}$。当我们谈论一个坐标系变换到另一个坐标系的问题时,问题中其实涉及了“$a$坐标系的$\vv{i}$、$\vv{j}$和$\vv{k}$”以及“$b$坐标系的$\vv{i}$、$\vv{j}$和$\vv{k}$”。更具体一点,在开始推导时,$\vv{OX}$是坐标系$e$的$\vv{i}$,而$\vv{ox}$是坐标系$b$的$\vv{i}$,但是后来获得$\vv{ox} =   \bf{R} \vv{i}$时,这里的$\vv{ox}$则是表达在坐标系$e$中。初看这会让人迷惑不解,但理解之后读者会发现这是比较不容易引起误解的描述方法。这样我们还可以将矩阵$\bf{R}$写成
$$
\bf{R} = \begin{bmatrix}
\vv{ox} & \vv{oy} & \vv{oz}\\
\end{bmatrix}
$$

因此,假设现在刚体上有一点$\vv{p}$,它在坐标系$b$中是$\vv{i}$、$\vv{j}$和$\vv{k}$的线性组合,在坐标系$e$中则是$\vv{ox}$、$\vv{oy}$和$\vv{oz}$的线性组合,假设组合的系数是$p_1$、$p_2$和$p_3$。我们用$\vv{p_e}$代表点$\vv{p}$在坐标系$e$中的矢量形式,用$\vv{p_b}$代表点$\vv{p}$在坐标系$b$中的矢量形式,则
$$
\vv{p_e} = p_1\vv{ox}+p_2\vv{oy}+p_3\vv{oz}
$$
$$
\vv{p_b} = p_1\vv{i}+p_2\vv{j}+p_3\vv{k}
$$
显然地
$$
\vv{p_e} = \bf{R}\vv{p_b}
$$

我们说矩阵$\bf{R}$将坐标系$b$中的点转换到坐标系$e$中,它是一个\textbf{旋转矩阵}。这一物理含义可以用下标来描述,记为$\bf{R}_{eb}$,所以最终我们写成
$$
\vv{p_e} = \bf{R}_{eb}\vv{p_b}
$$

\subsection{旋转矩阵的性质}
上一节我们获得的
$$
\bf{R} = \begin{bmatrix}
\vv{ox} & \vv{oy} & \vv{oz}\\
\end{bmatrix}
= \begin{bmatrix}
r_{11} & r_{12} & r_{13}\\
r_{21} & r_{22} & r_{23}\\
r_{31} & r_{32} & r_{33}
\end{bmatrix}
$$
是非常一般性的结论,任意坐标系之间都有相同的变换关系。

因为$\vv{ox}$是坐标系$b$的$\vv{i}$,所以它是单位向量,而且与同为单位向量的$\vv{oy}$和$\vv{oz}$都正交。因此我们说$\bf{R}$是一个单位正交矩阵(orthonormal matrix)。

我们很容易证明以下两个性质
$$
\bf{R}^T\bf{R} = \bf{R}\bf{R}^T = \bf{I}
$$
$$
\text{det}(R) = 1
$$

事实上,这两个性质是$\bf{R}$能够被称为笛卡尔坐标系之间变换的旋转矩阵的充要条件。所有这种旋转矩阵可以构成一个名为$SO(3)$的集合。$SO(3)$具有丰富的代数性质,本教程中不再详细讨论了。

值得一提的时,上述推导全部是在三维笛卡尔坐标系中进行的,相似的推导也可以在二维笛卡尔坐标系中进行。读者可以自行推导,以加深对于旋转矩阵性质的理解。
%%%%%%%%%%%%%%%%%%%%%%%%%%%%%%%%%%%%%%%%%%
%%%                                           有时间还可以再写                                  %%%%%%%%%
%%%%%%%%%%%%%%%%%%%%%%%%%%%%%%%%%%%%%%%%%%

\subsection{刚体变换}
了解旋转矩阵之后,我们回到飞行器在大地坐标系中飞行的场景。现在我们可以通过一个辅助坐标系来描述飞行器上某个质点相对与大地坐标系原点的位置关系了。考虑下图中的场景:
%%%%%%%%%%%%%插个图
%%%%%%%%%%%%%


\section{速度与角速度}
一个物体在空间中以恒定速度移动,同时进行了绕自身某个轴的旋转,这时物体上各点的速度各不相同。回想游乐园里的旋风旋转设施
\subsection{角速度的定义}	
\subsection{速度和角速度的关系}
$\vv{v} = \vv{\omega} \times  \vv{r}$
\subsection{角速度传感器}
在刚体的运动中,角速度是一个非常关键的状态量。在多旋翼飞行器的控制中,我们必须能够测量这个状态量,才能对它施以准确有效的控制。然而准确测量角速度并不容易,几十年来,角速度传感器都是人类技术水平的一个标尺。目前比较成熟的角速度传感器有以下几种。
\subsubsection{机械陀螺}
旋转的刚体会呈现一种反直觉的物理特性:如下图所示,一个绕$z$轴高速旋转的刚体,如果在$x$轴上施加一个力矩,它会开始绕$y$轴缓慢旋转。在youtube上我们可以看到很多使用车轮实现演示的视频。在完善了欧拉方程的推导之后,我们会在章节XXX中简要介绍这一现象的物理原理。

机械陀螺传感器利用了旋转刚体的这一特性,它的旋转刚体叫做陀螺。简单来说,如果按照上图中构成的传感器发生了绕$x$轴的转动,那么一定是因为$x$轴上有了外部施加的力矩。根据陀螺的$y$轴会发生的转动,我们可以进行反馈控制,给$x$轴施加一个反向力矩以消除陀螺的转动,施加的反向力矩等于外部施加的力矩,可以换算成角速度。

因为机械陀螺中有旋转的部件,并且涉及到反馈控制器,因此它的机械机构和软硬件系统都非常复杂。
\subsubsection{MEMS角速度传感器}
MEMS元器件是由大型集成电路的思想演变出来的新型机械制造工艺,在硅片上蚀刻、增加金属薄层,可以形成类似滑块、弹簧的结构。

如下图所示,MEMS角速度传感器包含一个或多个嵌在圆盘上的滑块,当传感器转动时,滑块会在向心力作用下发生移动,改变元器件整体的电容和电阻特性,因此能够用来检测向心力的大小,进而算出角速度的大小。
\subsubsection{光纤角速度传感器}
光纤角速度传感器利用了“赛格奈克效应”(Sagnac Effect),也即同一束光线分成两束分别穿过一个环形路线之后,整个环形路线的角速度就造成两束光线再汇合时出现相位差。这个原理在维基百科上有深入的讲解和推导。

光纤角速度传感器中包含光源、相位检测装置和光纤构成的线圈。将这些元件整合入小尺寸的传感器中比较昂贵,另外一个光纤线圈只能测量一个轴的角速度,因此需要三套线圈才能测量三轴的角速度。但是光纤角速度传感器精确度较高、没有机械部件所以寿命长,相比前述两种传感器有比较大的优势,在军事设备上应用较多。



很多资料中把角速度传感器都叫做“陀螺仪”,这种说法并不准确,因为除了机械陀螺仪中有旋转的陀螺以外,MEMS角速度传感器和光纤角速度传感器当中都不存在旋转的部件。

\subsection{角速度和旋转矩阵的关系}
\section{旋转矩阵的参数化}
\subsection{欧拉角}
在前述章节\ref{sec:bodyframe}中,我们在机体坐标系中介绍了横滚、俯仰、偏航三个轴的概念。
\subsection{四元数}
我们注意到上述定义中$\vv{i}\ \vv{j}=\vv{k}$。这是一个重要但是常常被忽视的关键定义。四元数有两种定义,在Hamilton自己最早的定义中$\vv{i}\ \vv{j}=\vv{k}$。但是后来在NASA JPL开始使用四元数之后,他们在定义中采用了$\vv{i}\ \vv{j}=-\vv{k}$。两种定义在最初的表达式中差别极小,但是给后续很多公式的推导造成了很大问题。很多人在查找四元数的相关资料时,会在不同的资料中看到同一个公式在不同的资料里恰巧差了一些正负号的尴尬情况。因此读者对于$\vv{i}$和$\vv{j}$相乘的结果要格外小心。本篇教程中的四元数采用Hamilton的古典定义。

确定四元数的定义之后,我们可以开始介绍四元数的乘法:
四元数求逆:

如果四元数满足:
这种四元数则被称为旋转四元数。
四元数的矢量部自然就表示了旋转的轴和旋转的量

旋转四元数和旋转矩阵一样可以用来表示
\section{欧拉方程}
​根据牛顿第二定律,对于刚体的位置、速度、加速度来说,给定外力和初始状态以后,系统的状态是由二阶常微分方程确定的。

对应地,对于绕定轴旋转的刚体的角度、角速度和角加速度来说,则是欧拉方程决定了系统的状态,它同样也是二阶常微分方程。相比于牛顿第二定律来说,由于旋转导致了坐标系不断变化,因此在方程中需要考虑角速度对系统的状态的影响。本节将会通过介绍角动量和惯性张量,并结合角速度的特性推导欧拉方程
\subsection{质点角动量}
\subsection{刚体角动量和惯性张量}
\subsection{欧拉方程与多旋翼飞行器}
在多旋翼飞行器的控制中,欧拉方程是最核心的公式。因为多旋翼飞行器的控制本质就是通过力矩和欧拉方程包含的二阶微分方程控制姿态角。多旋翼飞行器在空中飞行的时候,通过调整自己的姿态角来产生往某个方向的推力。系统的牛顿方程中,F的x、y分量实际上是由姿态角决定的。在最基本的多旋翼飞行器控制器中,通过改变电机转速控制好三个轴的力矩就可以保证飞行器在空中保持姿态稳定。
\section{通过仿真程序理解前述的知识}
本教程只针对多旋翼飞行器中的核心动力学建模问题进行讲解,为了让飞行器能够在空中实现姿态稳定,还需要能够测量角速度、解算姿态角、计算电机转速和三轴力矩之间的关系、设计姿态控制器。这些问题超出了本教程的讨论范围。

在中有一个简单的四旋翼飞行器模拟器。在RigidBody.cpp中我们展示了如何通过程序实现欧拉方程的仿真。

读者可以利用RigidBody.cpp和相关的辅助代码自行通过仿真程序理解欧拉方程。


\end{document}
